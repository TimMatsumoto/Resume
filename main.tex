
\documentclass[11pt,a4paper,sans]{moderncv}  
\moderncvstyle{banking}                            % style options are 'casual' (default), 'classic', 'oldstyle' and 'banking'
\moderncvcolor{blue}                                % color o

% adjust the page margins
\usepackage[scale=0.90, top=10mm]{geometry}

\usepackage{import}

% personal data
\name{Tim (Tetsuya)}{Matsumoto}

\email{t.matsumoto@alumni.ubc.ca}                               % optional, remove / comment the line if not wanted
\homepage{github.com/TimMatsumoto}    
% \homepage{www.linkedin.com/in/tetsuya-tim-matsumoto-4a0b32122}    


\begin{document}

\makecvtitle
\vspace*{-10mm}

\section{Education}


    \cventry    
        {September 2016 - Ongoing}
        {Bachelor of Science, Honours in Mathematics with minor in Computer Science}
        {University of British Columbia}    
        {}{}
        {\vspace{3pt} Expected May 2020 | Vancouver, B.C, Canada | Major GPA 3.90/4.33}

    \vspace{6pt}

    \cventry
        {August 2015 - January 2017} 
        {Student of Mathematics}
        {Stanford Pre-Collegiate University-Level Online Math \& Physics}   
        {}{}{}



\section{Skills}


    \cvitem{
        \textbf{Programming Languages}}
            {Python, C++, Matlab, Java, C\#, TeX, XML }   


    \cvitem{
        \textbf{Industry Skills}} 
            {Git, OpenGL, Unity, IDE tools and libraries.}

    \cvitem{
        \textbf{Academic Knowledge}} 
            {Machine Learning/Deep Learning, 3D graphics, Linear Algebra, 
            Numerical Operations, Differential Equation, Differential Geometry}


\section{Projects}


    \vspace{5pt}


    \cventry
        % Current Status
        {August 2018 - Ongoing} 
        % Language/Library
        {C\#}
        % Name
        {4D SLASH}
        % Category
        {Unity3D}
        % Tools/Environment
        {Unity/Visual Studio}
        % Description
        \hline
        {  
            A 3D game in Unity3D. A player moves along a predefined trajectory in 3D space
        avoiding obstacles on the way. The player also has a blade to slay and cut 
        the obstacles. Some obstacles are thrown targeted at the player, and the player 
        cuts the obstacle to avoid crashing. There are also objects from which the player 
        gets some kind of rewards. As extra features, I may add more assets to play with 
        such as guns, beams Japanese sword and so on. 
        The trajectories will be implemented in C++ and combined into C\# file.
        }

    \noindent\rule{\textwidth}{0.1pt}
        
        \vspace{3pt}

    % \cventry
    %     % Current Status
    %     {August 2018}
    %     % Language/Library
    %     {Java/XML}
    %     % Name
    %     {AutoCutBlue}
    %     % Category
    %     {Android Application} 
    %     % Tools/Environment
    %     {IntelliJ/Android Studio}
    %     % Description
    %     \hline
    %     {
    %         A Bluetooth app that sets timer to disconnect devices and save battery.
    %     This android app disconnects Bluetooth connection after certain amount of time 
    %     set by the user. It prevents draining battery by playing music when the user
    %     is not actually listening.
    %     }


    % \noindent\rule{\textwidth}{0.1pt}

    % \vspace{6pt}

    \cventry    
        % Current Status
        {August 2018}
        % Language/Library
        {Python/keras}
        % Name
        {Stock Price Prediction}
        % Category
        {Deep Learning}
        % Tools/Environment
        {}
        % Description
        \hline
        {
            Predict Google stock price for next 60 business days from past data using
        RNN with keras library. Construct a data structure from data set from Google homepage into a csv file 
        where $k_{th}$ row represents the stock prices from the day to 20 days prior 
        to the day. Then, Recurrent Neural Network (RNN) is applied to learn the data set,
        and the result is plotted with the actual changes in the stock price for comparison.
        This is implemented in Python using Keras library.
        }

        \noindent\rule{\textwidth}{0.1pt}

    \vspace{3pt}

    \cventry
        % Current Status
        {August 2018}
        % Language/Library
        {Python/Pytorch}
        % Name
        {Movie Recommendation}
        % Category
        {Deep Learning}
        % Tools/Environment
        {}
        % Description
        \hline
        {
            A movie recommendation system using Restricted Boltzmann Machine (RBM) with Pytorch library. 
        Algorithm 2 from "An Introduction to Restricted Boltzmann Machines" by A. Fischer and C. Igel 
        is implemented using Pytorch library. Although RBM is a unsupervised learning algorithm, 
        the data set was divided into training set and test set to test the RBM classifier. 
        The final result was approximately 75 \% in accuracy.
        
        }


%     \vspace{5pt}

% \section{Academic Projects}

\noindent\rule{\textwidth}{0.1pt}

    \vspace{3pt}
           
        \cventry
            % Current Status
            {May - June 2018}
            % Language/Library
            {C++}   
            % Name
            {Blurring Tree}
            % Category
            {Algorithms and Data Structures}
            % Tools/Environment
            {}
            % Description
            \hline
            {
                A tree is constructed from an image to blur. In prior to tree construction, 
                for every pixel, the average and variance of RGB colours in the rectangle 
                where the upper left corner is (0,0) of the image and lower right corner 
                is the pixel. Then, The tree splits either horizontally or vertically based 
                on the statistics. This process continues until the leaf node corresponds to 
                each pixel. By pruning this tree, we get a pruned image.
            }

        % \noindent\rule{\textwidth}{0.1pt}

        % \vspace{6pt}

        % \cventry
        %     % Current Status
        %     {September - December 2017}
        %     % Language/Library
        %     {Java}
        %     % Name
        %     {Bus R Us}
        %     % Category
        %     { Software Construction}
        %     % Tools/Environment
        %     {JSON/Translink Open API}
        %     % Description
        %     \hline
        %     {
        %         A real time bus application on android devices. 
        %     This application maps the location of stops, buses and bus routes on the 
        %     Greater Vancouver Transit system (Translink). It retrieves real time arrival 
        %     information at each stops. The obtained JSON files which contain bus stop 
        %     information and bus route information are parsed and stored in a hash table.
        %     Bus routes and icons for bus stops and vehicles are rendered on the map. 
        %     }



\end{document}


